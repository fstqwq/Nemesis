\paragraph{弦图的定义}连接环中不相邻的两个点的边称为弦. 一个无向图称为弦图, 当图中任意长度都大于 $3$ 的环都至少有一个弦.
\paragraph{单纯点}一个点称为单纯点当 $\{v\}\cup A(v)$ 的导出子图为一个团. 任何一个弦图都至少有一个单纯点, 不是完全图的弦图至少有两个不相邻的单纯点. 
\paragraph{完美消除序列}一个序列 ${v_1,v_2,...,v_n}$ 满足 $v_i$ 在 ${v_i,\cdots,v_n}$ 的诱导子图中为一个单纯点.
一个无向图是弦图当且仅当它有一个完美消除序列.
\paragraph{最大势算法} 从 $n$ 到 $1$ 的顺序依次给点标号. 设 $\mathrm{label}_i$ 表示第 $i$ 个点与多少个已标号的点相邻, 每次选择 $\mathrm{label}$ 最大的未标号的点进行标号. 用桶维护优先队列可以做到 $O(n + m)$. 
\paragraph{弦图的判定} 判定最大势算法输出是否合法即可. 如果依次判断是否构成团, 时间复杂度为 $O(nm)$.
考虑优化, 设 ${v_{i+1},\cdots,v_n}$ 中所有与 $v_i$ 相邻的点依次为 $N(v_i) = \{v_{j1},\cdots,v_{jk}\}$. 
只需判断 $v_{j1}$ 是否与 $v_{j2},\cdots,v_{jk}$ 相邻即可. 时间复杂度$O(n+m)$.
\paragraph{弦图的染色} 完美消除序列\textbf{从后往前}染色, 染上出度的 mex.
\paragraph{最大独立集} 完美消除序列\textbf{从前往后}能选就选.
\paragraph{团数} 最大团的点数. 一般图团数 $\leq$ 色数, 弦图团数 = 色数. 
\paragraph{极大团} 弦图的极大团一定为 $\{x\} \cup N(x)$.
\paragraph{最小团覆盖} 用最少的团覆盖所有的点. 设最大独立集为 $\{p_1, \dots ,p_t\}$, 
则 $\{p_1\cup N(p_1), \dots , p_t \cup N(p_t)\}$ 为最小团覆盖.
\paragraph{弦图 $k$ 染色计数} $\prod_{v\in V} k - N(v) + 1$.
\paragraph{区间图} 每个顶点代表一个区间, 有边当且仅当区间有交. 区间图是弦图, 一个完美消除序列是右端点排序.
% \paragraph{区间图的判定} 本质上就是一个 $01$ 矩阵, 通过交换列来使得 $1$ 连续.