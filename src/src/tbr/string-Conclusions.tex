\subsubsection*{双回文串}
\noindent
如果$s=x_1x_2=y_1y_2=z_1z_2, |x_1|<|y_1|<|z_1|, x_2, y_1, y_2, z_1$是回文串, 则$x_1$和$z_2$也是回文串. 

\subsubsection*{Border 和周期}
\noindent
如果$r$是$S$的一个border, 则$|S|-r$是$S$的一个周期.\\
如果$p$和$q$都是$S$的周期, 且满足$p+q\le |S|+gcd(p,q)$, 则$gcd(p,q)$也是一个周期.

\subsubsection*{字符串匹配与Border}
\noindent
若字符串$S$, $T$满足$2|S|\ge |T|$, 则$S$在$T$中所有匹配位置成等差数列.\\
若$S$的匹配次数大于2, 则等差数列的周期恰好等于$S$的最小周期.

\subsubsection*{Border 的结构}
\noindent
字符串$S$的所有不小于$|S|/2$的border长度组成一个等差数列.\\
字符串$S$的所有 border 按长度排序后可分成$O(\log{|S|})$段, 每段是一个等差数列. 

\subsubsection*{回文串Border}
\noindent
回文串长度为$t$的后缀是一个回文后缀, 等价于$t$是该串的border. 因此回文后缀的长度也可以划分成$O(\log{|S|})$段.

\subsubsection*{子串最小后缀}
\noindent
设$s[p..n]$是$s[i..n]$, $(l \leq i \leq r)$中最小者, 则minsuf(l, r)等于$s[p..r]$的最短非空 border. minsuf(l, r) = min\{$s[p..r]$, minsuf(r − $2^k$ + 1, r)\}, $(2^k < r − l + 1 \leq 2^{k+1})$. 

\subsubsection*{子串最大后缀}
\noindent
从左往右扫, 用set维护后缀的字典序递减的单调队列, 并在对应时刻添加"小于事件"点以便在之后修改队列; 查询直接在set里lower\_bound. 

\subsubsection*{ZJJ: SAM处理手法}
\noindent 1. 基本子串结构: CLB 搞的那玩意。\\	
\noindent 2. 正反串 SAM 的基本联系: 一个子串出现的位置将会在两个SAM中同时得到映照。\\	
\noindent 3. SAM 上转成数点问题。\\	
\noindent 4. 线段树合并维护 endpos 集合。\\	
\noindent 5. 树剖保证到根的链上只涉及 log 次修改和查询。(区间 border) \\	
\noindent 6. LCT 保证到根的链只修改均摊 log 个不同的颜色段。(区间本质不同子串数量)

\subsubsection*{ZJJ: 字符串常见错误}
\noindent 1. 字符串算法变式记得判匹配位置超出字符串的情况,例如多组数据下的双端插入回文串,后缀数组多组。
\noindent 2. 警惕 char 运算中 `a' 和 `\textvisiblespace{a}' 的区别。
\noindent 3. char kmp[]